\documentclass[a4paper,12pt]{article}
\usepackage{makeidx}
\usepackage[top=3cm, bottom=1cm, left=1cm, right=1cm]{geometry}
\usepackage{fancyhdr}
\pagestyle{fancy}
\usepackage{xstring}
\usepackage{ifthen}
\usepackage[portuguese]{babel}
\fancyhf{}
\fancyhead[C]{\tiny CONTINUAÇÃO DO CONTRATO DE LOCAÇÃO DO IMÓVEL SITUADO NA ${dbg.enderecoC}$, APT. Nº ${db.apartamento}$ – PORTO DE GALINHAS/IPOJUCA (PE) CEP: 55.590-000\\}
\date{\today}
\author{ENRIQUE ALBERTO DEL CESARE}
\title{CONTRATO DE LOCAÇÃO DE IMÓVEL}
\begin{document}
\thispagestyle{empty}
\begin{center}{\bf \huge CONTRATO DE LOCAÇÃO DE IMÓVEL}\\[5.1cm] \end{center} 
\newcommand\n{\newcommand}


Os signatários deste  instrumento de um lado Enrique Alberto Del Cesare,
argentino, naturalizado brasileiro, solteiro, CPF nº 578.550.724-20, identidade nº 2.583.252 SDS\//PE, 
residente a Rua Cavalo Marinho, nº 180 Recanto Porto de Galinhas II Quadra 5 Lote G1 
no bairro de Porto de Galinhas Ipojuca/Pernambuco CEP 55.590-000 e, de outro,
${db.nome}$, ${db.nacionalidade}$, ${db.estadoCivil}$, ${db.profissao}$, CPF nº ${db.cpf}$, identidade ${db.identidade}$, 
residente a ${db.endereco}$. 

O primeiro nomeado, aqui designado “LOCADOR”, sendo proprietário do imóvel situado à rua ${dbg.enderecoc}$  
no bairro de Porto de Galinhas Ipojuca\//Pernambuco, CEP 55.590-000 loca-o ao segundo, aqui designado “LOCATÁRIO”, mediante as cláusulas e condições seguintes:

\begin{enumerate}

\item O prazo do presente contrato é de \textbf{01 (hum) ano} a iniciar em ${db.data}$
\ e a terminar em ${dbg.dataFinal}$\ data em que o LOCATÁRIO 
se obriga a restituir o imóvel completamente desocupado, nas condições previstas neste contrato,
sob pena de incorrer na multa da cláusula 9º e de sujeitar-se ao disposto no Artigo 1196 do Código Civil Brasileiro;
\item O aluguel mensal é de \textbf{R\$ ${db.aluguel}$,00 (${dbg.aluguelExtenso}$ reais)},
que o LOCATÁRIO se compromete a pagar pontualmente, até o dia ${db.vencimento}$\
de cada mês, na residência do LOCADOR ou efetuando o depósito no Banco do Brasil Agência 4600-0 poupança número 9457-9, operação 51; 
\item Os consumos de \textbf{água e luz}, assim como taxas de condomínio, que incidam ou venham a incidir sobre o imóvel, conservação e outras decorrentes de lei, assim como suas respectivas majorações, \textbf{ficam a cargo do LOCATÁRIO} e, seu não pagamento na época determinada, acarretará a rescisão deste; 
\item O LOCATÁRIO, salvo as obras que importem na segurança do imóvel, obriga-se por todas as outras, devendo trazer o imóvel locado em boas condições de higiene e limpeza, e pinturas, em perfeito estado de conservação e funcionamento, para assim restituí-los quando findo ou rescindido este contrato, sem direito a retenção ou indenização por quaisquer benfeitorias ainda que necessárias, as quais ficarão desde logo incorporadas ao imóvel; 
\item Obriga-se o LOCATÁRIO no curso da locação, a satisfazer a todas as exigências dos Poderes Públicos a que der causa, não motivando elas a rescisão deste contrato;  
\item Não é permitida a transferência deste contrato, nem a sub-locação, cessão ou empréstimo total ou parcial do imóvel, sem prévio consentimento por escrito do LOCADOR, devendo no caso deste ser dado, agir oportunamente junto aos ocupantes, afim de que o imóvel esteja desimpedido nos termos do presente contrato. Igualmente não é permitido fazer modificações no imóvel, sem autorização do LOCADOR;
\item O LOCATÁRIO desde já faculta ao LOCADOR, examinar ou vistoriar o imóvel locado quando entender conveniente;
\item Fica estipulada a multa de 10\% (Dez por Cento) para pagamento fora do vencimento, na qual incorrerá a parte que infringir qualquer cláusula deste contrato, com a faculdade, para a parte inocente, de poder considerar simultaneamente rescindida a locação, independentemente de qualquer formalidade;
\item Para todas as questões oriundas deste contrato, será competente o foro da situação do imóvel, com renúncia de qualquer outro, por mais especial que se apresente;
\item Tudo quanto for devido em razão deste contrato e, que não comportem o processo executivo, será cobrado em ação competente, ficando a cargo do devedor, em qualquer caso, os honorários do advogado que o credor constituir para ressalva dos seus direitos;
\item Quaisquer estragos ocasionados ao imóvel e suas instalações, bem como as despesas a que o proprietário for obrigado por eventuais modificações feitas no imóvel, pelo LOCATÁRIO, não ficam compreendidas na multa da cláusula 9ª , mas serão pagas à parte;
\item O imóvel, objeto desta locação, destina-se exclusivamente para fins residenciais, 
reservado para \textbf{ ${dbg.limitePessoas}$ pessoas },  não podendo, a sua destinação ser mudada sem o consentimento do LOCADOR;
Será cobrado uma multa de 100 reais por dia, por pessoa além do limite;
\item Quando da rescisão do contrato pelo LOCATÁRIO  ou pelo LOCADOR, o desistente será penalizado com uma multa no valor correspondente a 2 (dois) meses de aluguel. O LOCADOR não será penalizado quando o motivo da desistência for motivado pela não observância de qualquer uma das cláusulas pelo LOCATÁRIO;
\item Os objetos citados nesta cláusula: ${db.objetos}$;
\item Demais regras de convivência no condomínio: Proibido som alto, jogar lixo no chão ou jardim, como cigarros, latinhas etc. 
Estender pranchas ou outros objetos na área externa.
Proibido festas, piscinas de plástico, lavar carros ou bicicletas no condomínio.
${dbg.bicicletas}$ ${dbg.animais}$ ${dbg.garagem}$

\end{enumerate}
Assim, por estarem justos e contratados, assinam o presente, em 2 (duas) vias de igual teor.

\begin{flushright}
Ipojuca, \today
\end{flushright}

\begin{center}

\vspace{2cm}

LOCADOR: \hrulefill
\\Enrique Alberto Del Cesare

\vspace{2.5cm}

LOCATÁRIO: \hrulefill
\\${db.nome}$
 \end{center}

\end{document}
